%----------------------------------------------------------------------------%
% Autores: Elias C. Alves,
% NextStep Empresa Júnior de Sistemas de Informação
% Licenciado sobre o GNU GENERAL PUBLIC LICENSE Version 3
% unixelias@gmail.com

\documentclass[a4paper, 12pt]{article}
\renewcommand{\familydefault}{\sfdefault}
\usepackage{helvet}

\usepackage{fullpage}
\usepackage[utf8]{inputenc}
\usepackage[brazil]{babel}

\usepackage{graphicx}
\usepackage{tabularx}
\usepackage{booktabs}
\usepackage{fancyref}
\usepackage{fancyhdr}

%----------------------------------------------------------------------------%
% Variáveis referentes ao projeto e ao documento
\newcommand\semana{11/12 - 17/12/2016}
\newcommand\datareuniao{15/12/2016}
%\newcommand\datareuniaob{17/12/2016}
\newcommand\equipe{Equipe Kepler}
\newcommand\projeto{Projeto BrasilTur}
\newcommand\semestre{02 - 2016}
\newcommand\assunto{Relatório de reunião com o cliente}
\newcommand\palavraschave{controle do tempo; protocolo; trabalho; tarefas}
\newcommand\nome{\projeto, \equipe, \semestre}
\newcommand\titulo{Reunião com o Cliente em \datareuniao}
%----------------------------------------------------------------------------%

%----------------------------------------------------------------------------%
% Insere metadados no PDF
\usepackage[pdftex, pdfcreator={TeXstudio}]{hyperref}
\hypersetup{
		pdfauthor={\nome},
		pdftitle={\titulo},
		pdfsubject={\assunto},
		pdfkeywords={\palavraschave},
		pdfproducer={\equipe},
}

\pagestyle{fancy}
\fancyhead{}
\fancyfoot{}
\fancyfoot[C]{\thepage}
\renewcommand{\headrulewidth}{0pt}
%----------------------------------------------------------------------------%

\begin{document}

%----------------------------------------------------------------------------%
% Página de Título
\begin{titlepage}

\begin{center}
\includegraphics[width=\linewidth]{../../images/next_step_logo.png} \\[0.8cm]
\textsc{\LARGE NextStep - Empresa Júnior} \\[0.2cm]
\textsc{\LARGE Sistemas de Informação - UFVJM} \\[0.2cm]
\textsc{\LARGE Diretoria de Projetos} \\[0.2cm]
\textsc{\equipe} \\[1.3cm]
\vfill
%Descomentar para incluir a logo do projeto
\includegraphics[width=0.3\linewidth]{../../../img/brtur1.png} \\[0.8cm]
\textsc{\LARGE \projeto} \\[0.8cm]
\vfill
\vfill
{\LARGE \titulo \\[0.4cm]}
\includegraphics[width=\linewidth]{../../images/jc_rodape.png} \\[0.8cm]
\end{center}
\end{titlepage}
\clearpage
%----------------------------------------------------------------------------%
% Início de subseção
%----------------------------------------------------------------------------%
\subsection*{Relatório de reunião com cliente}

%----------------------------------------------------------------------------%

\subsubsection*{Pauta}
\begin{itemize}
	\item Repasses sobre o projeto;
	\item Novo prazo para entrega;
	\item Funções para emissão de boletos;
	\item Geração de relatórios;
\end{itemize}

\subsubsection*{Participantes da reunião}
\begin{itemize}
	\item Elias da Cunha Alves - Diretor Interino de Projetos
	\item Paulo Henrique Cruz - Cliente
\end{itemize}

\subsubsection*{Repasses sobre o projeto}
O cliente foi atualizado sobre o estado do projeto e foi informado do atraso decorrente da greve docente da UFVJM, que fez com que a equipe se dispersasse, dificultando o processo de desenvolvimento. Além disso, houveram algumas dúvidas de ordem técnica a respeito do processo de emissão de boletos.

\subsubsection*{Novo prazo para entrega}
O Cliente relatou insatisfação com os atrasos e comentou que pensou em cancelar o contrato.
Contudo, disse que precisa do sistema funcionando com certa urgência e disse que entende as dificuldades da empresa júnior, mas espera que o projeto seja finalizado em breve.

\subsubsection*{Processo de emissão de boletos}
O cliente precisa de dois tipo de boletos:
\begin{itemize}
	\item Boleto único em Formato A4.
	\item Carnê de prestações com vários boletos, em Formato A4, com capa com informações do sacado e 3 boletos por folha com canhoto.
\end{itemize}

Os arquivos de remessa bancária deverão ser baixados pelo cliente e enviados ao banco, enquanto que os de retorno devem ser inseridos no sistema pelos usuários.
O Cliente desejava um sistema que fizesse isso de maneira automática, mas o SICOOB informou que isso não pode ser feito, por limitações do sistema deles.
 
\subsubsection*{Relatórios}
O Cliente deseja ver a parte de relatórios funcionando e listou alguns dos que deseja. Dentre outros, relatórios com diferenciação entre clientes maiores e menores de idade, pois nem sempre quem contrata é quem viaja. Então seria necessário uma diferenciação entre o contratante e o passageiro.
O mesmo se aplica para a geração de contratos. O Cliente deseja a emissão automática de contratos, para tal, também é necessário fazer essa diferenciação entre contratante e passageiro. Sendo assim ele disse que pode ser o nome de apenas um contratante e em outro campo, uma listagem dos passageiros. Junto com o contrato, deve ser gerada uma autorização para menores de idade.

\subsubsection*{Outros assuntos}
O Cliente repassou arquivos reais com o layouts de emissão e retorno bancários, além dos modelos de contrato e outras informações, que serão compartilhadas com os membros da equipe.
O Cliente pediu um retorno sobre a hospedagem, que segundo ele, ficou a cargo do André, Analista da Next Step.

\subsubsection*{Observações}
O Cliente solicitou a opção de o cliente fazer o pagamento de diferentes formas, assim, o valor de um contrato poderia ser pago parte com cheque e parte com boleto, por exemplo.

\clearpage
% Fim de Subseção
%----------------------------------------------------------------------------%


\end{document}
