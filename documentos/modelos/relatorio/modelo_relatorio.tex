%----------------------------------------------------------------------------%
% Autores: Elias C. Alves,
% NextStep Empresa Júnior de Sistemas de Informação
% Licenciado sobre o GNU GENERAL PUBLIC LICENSE Version 3
% unixelias@gmail.com

\documentclass[a4paper, 12pt]{article}
\renewcommand{\familydefault}{\sfdefault}
\usepackage{helvet}

\usepackage{fullpage}
\usepackage[utf8]{inputenc}
\usepackage[brazil]{babel}

\usepackage{graphicx}
\usepackage{tabularx}
\usepackage{booktabs}
\usepackage{fancyref}
\usepackage{fancyhdr}

%----------------------------------------------------------------------------%
% Variáveis referentes ao projeto e ao documento
\newcommand\semana{12/06 - 18/06/2016}
\newcommand\datareuniao{15/06/2016}
\newcommand\equipe{Equipe de Pesquisa Joan Clarke}
\newcommand\projeto{Ferramentas de Compartilhamento de Código}
\newcommand\semestre{01 - 2016}
\newcommand\assunto{Relatório de pesquisa desenvolvida}
\newcommand\palavraschave{controle do tempo; protocolo; trabalho; tarefas}
\newcommand\nome{\projeto, \equipe, \semestre}
\newcommand\titulo{Relatório de Pesquisa}
%----------------------------------------------------------------------------%

%----------------------------------------------------------------------------%
% Insere metadados no PDF
\usepackage[pdftex, pdfcreator={TeXstudio}]{hyperref}
\hypersetup{
		pdfauthor={\nome},
		pdftitle={\titulo},
		pdfsubject={\assunto},
		pdfkeywords={\palavraschave},
		pdfproducer={\equipe},
}

\pagestyle{fancy}
\fancyhead{}
\fancyfoot{}
\fancyfoot[C]{\thepage}
\renewcommand{\headrulewidth}{0pt}
%----------------------------------------------------------------------------%

\begin{document}

%----------------------------------------------------------------------------%
% Página de Título
\begin{titlepage}

\begin{center}
\includegraphics[width=\linewidth]{../images/next_step_logo.png} \\[0.8cm]
\textsc{\LARGE NextStep - Empresa Júnior} \\[0.2cm]
\textsc{\LARGE Sistemas de Informação - UFVJM} \\[0.2cm]
\textsc{\LARGE Diretoria de Projetos} \\[0.2cm]
\textsc{\equipe} \\[1.3cm]
\vfill
%Descomentar para incluir a logo do projeto
%\includegraphics[width=0.6\linewidth]{images/imagem.png} \\[0.8cm]
\textsc{\LARGE \projeto} \\[0.8cm]
\vfill
\vfill
{\LARGE \titulo \\[0.4cm]}
\includegraphics[width=\linewidth]{../images/jc_rodape.png} \\[0.8cm]
\end{center}
\end{titlepage}
\clearpage
%----------------------------------------------------------------------------%

%----------------------------------------------------------------------------%
% Descrição das ferramentas pesquisadas.
% * Dividimos em seções por assunto e em cada assunto, uma subseção descreve 
%   a ferramenta a ser apresentada.
% * Existem alguns campos como exemplo, mas podemos acrescentar outros se
%   necessário
%----------------------------------------------------------------------------%

%----------------------------------------------------------------------------%
% Início de seção
\section{Assunto}

%----------------------------------------------------------------------------%
% Início de subseção
\subsection{Nome da Ferramenta}

\subsubsection{Descrição}
Breve descrição da ferramenta

\subsubsection{Formato de dados}
Descrição do formato dos dados, quantidade de dados

\subsubsection{Limite de usuários}
Descrever como é tratado o limite de usuários da ferramenta

\subsubsection{API}
Descrição das APIs da ferramenta

\subsubsection{Extensões}
Descrição dos plugins e/ou extensões de interesse

\subsubsection{Licença}
Descrição da licença da ferramenta

\subsubsection{Hospedagem e segurança de dados}
Descrição das implicações da política de segurança da informação e política de privacidade dos dados

\subsubsection{Observações}
Observações, se necessário

\clearpage
% Fim de Subseção
%----------------------------------------------------------------------------%

%----------------------------------------------------------------------------%
% Início de subseção
\subsection{Nome da Ferramenta}

\subsubsection{Descrição}
Breve descrição da ferramenta

\subsubsection{Formato de dados}
Descrição do formato dos dados, quantidade de dados

\subsubsection{Limite de usuários}
Descrever como é tratado o limite de usuários da ferramenta

\subsubsection{API}
Descrição das APIs da ferramenta

\subsubsection{Extensões}
Descrição dos plugins e/ou extensões de interesse

\subsubsection{Licença}
Descrição da licença da ferramenta

\subsubsection{Hospedagem e segurança de dados}
Descrição das implicações da política de segurança da informação e política de privacidade dos dados

\subsubsection{Observações}
Observações, se necessário

\clearpage
% Fim de Subseção
%----------------------------------------------------------------------------%
% Fim de Seção
%----------------------------------------------------------------------------%


%----------------------------------------------------------------------------%
% Início de seção
\section{Assunto seguinte}

%----------------------------------------------------------------------------%
% Início de subseção
\subsection{Nome da Ferramenta}

\subsubsection{Descrição}
Breve descrição da ferramenta

\subsubsection{Formato de dados}
Descrição do formato dos dados, quantidade de dados

\subsubsection{Limite de usuários}
Descrever como é tratado o limite de usuários da ferramenta

\subsubsection{API}
Descrição das APIs da ferramenta

\subsubsection{Extensões}
Descrição dos plugins e/ou extensões de interesse

\subsubsection{Licença}
Descrição da licença da ferramenta

\subsubsection{Hospedagem e segurança de dados}
Descrição das implicações da política de segurança da informação e política de privacidade dos dados

\subsubsection{Observações}
Observações, se necessário

\clearpage
% Fim de Subseção
%----------------------------------------------------------------------------%
% Fim de Seção
%----------------------------------------------------------------------------%


\end{document}
